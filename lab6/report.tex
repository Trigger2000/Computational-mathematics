\documentclass[a4paper, 12pt]{article}

\usepackage[T2A]{fontenc}
\usepackage[utf8]{inputenc}
\usepackage[english,russian]{babel} %локализация
\usepackage[left=2cm,right=2cm, top=2cm,bottom=2cm, bindingoffset=0cm] {geometry}
\usepackage{graphicx}%Вставка картинок правильная
\usepackage{float}%"Плавающие" картинки
\usepackage{wrapfig}%Обтекание фигур (таблиц, картинок и прочего)
\usepackage{fancybox,fancyhdr} %колонтитулы
\usepackage{amsmath, amsfonts, amssymb, amsthm, mathtools} %математика

%\usepackage[12pt]{extsizes} % размер шрифта

\begin{document}
	\begin{center}
		\textbf{Отчет по задаче 4.1.12 (Демченко). Трубачев Илья}
	\end{center}

	\begin{equation*}
		\frac{d}{dx} [k(x) \frac{du}{dx}] - q(x)u = -f(x)
	\end{equation*}
	
	\begin{equation*}
	\begin{cases}
	k(0) u_x(o) = u(0) - 1 \\
	-k(1) u_x(1) = u(1)
	\end{cases} \\
	0 \leq x \leq 1, h = 0.1
	\end{equation*}
	
	
	В файле plot\_model.svg график для модельной задачи $k(x) = \sqrt{e}, q(x) = \sqrt{e}, f(x) = cos(0.5)$. Численное решение модельной задачи совпадает с аналитическим решением. В файле plot\_task.svg для задачи $k(x) = e^x, q(x) = e^x, f(x) = cosx$.
\end{document}