\documentclass[a4paper, 12pt]{article}

\usepackage[T2A]{fontenc}
\usepackage[utf8]{inputenc}
\usepackage[english,russian]{babel} %локализация
\usepackage[left=2cm,right=2cm, top=2cm,bottom=2cm, bindingoffset=0cm] {geometry}
\usepackage{graphicx}%Вставка картинок правильная
\usepackage{float}%"Плавающие" картинки
\usepackage{wrapfig}%Обтекание фигур (таблиц, картинок и прочего)
\usepackage{fancybox,fancyhdr} %колонтитулы
\usepackage{amsmath, amsfonts, amssymb, amsthm, mathtools} %математика

%\usepackage[12pt]{extsizes} % размер шрифта

\begin{document}
	\begin{center}
		\textbf{Отчет по задаче 8.11.4. Трубачев Илья}
	\end{center}

	\begin{equation*}
	\begin{cases}
	x' = z \\
	y' = u \\
	z' = -\frac{x}{(x^2 + y^2)^{3/2}} \\
	u' = -\frac{y}{(x^2 + y^2)^{3/2}}
	\end{cases} \\
	0 < t < 20, x(0) = 0.5, y(0) = z(0) = 0, u(0) = 1.73
	\end{equation*}
	
	Использованы методы Рунге-Кутты с соответствующими таблицами Бутчера:
	\begin{center}
	\begin{tabular}{|c|c|}
		\hline
		0 & 0 \\
		\hline
		& 1 \\
		\hline
	\end{tabular}
	\bigskip
	\begin{tabular}{|c|c|c|}
		\hline
		0 &  &  \\
		\hline
		1/2 & 1/2 &  \\
		\hline
		& 0 & 1 \\
		\hline
	\end{tabular}
	\bigskip
	\begin{tabular}{|c|c|c|c|}
		\hline
		0 &  &  &  \\
		\hline
		1/2 & 1/2 &  &  \\
		\hline
		1 & 0 & 1 &  \\
		\hline
		& 1/6 & 4/6 & 1/6 \\
		\hline
	\end{tabular}
	\bigskip
	\begin{tabular}{|c|c|c|c|c|}
		\hline
		0 &  &  &  &  \\
		\hline
		1/2 & 1/2 &  &  &  \\
		\hline
		1/2 & 0 & 1/2 &  &  \\
		\hline
		1 & 0 & 0 & 1 &  \\
		\hline
		& 1/6 & 2/6 & 2/6 & 1/6 \\
		\hline
	\end{tabular}
	\end{center}
	
	Построены графики u(t) и траектория f(x, y, z) = 0 для различных значений шага интегрирования. Видно, что точность построения графиков зависит от шага интегрирования: для наглядности можно сравнить графики, построенные методом РК первого порядка для шага 0.5 и 0.001. 
\end{document}